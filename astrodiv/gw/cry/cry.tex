\subsubsection*{\bf Cryogenic system}
\noindent {\sf [Spokesperson :\ Takafumi USHIBA]}

\noindent {\sf \small ICRR, The Univ.\ of Tokyo, Hida, Gifu 506-1205}
\vspace{3pt}

One of unique features of KAGRA is cooling sapphire mirrors, which are installed at 3-km arm cavity. Members working for KAGRA cryogenic system, which plays an important role for achieving this unique characteristics, are mainly constituted by that of ICRR, KEK, and the University of Toyama. Here, we summarize the activity of members of ICRR in FY 2019.

\paragraph*{\bi Cryogenic payload}
A KAGRA sapphire mirror is suspended by 9-stage suspension and its bottom 4 stages that include sapphire mirror is called cryogenic payload. Angular and translational motion of these sapphire mirrors must be controlled well for operating them as an interferometer. Therefore, damping and global control of suspension were implemented and improved in this fiscal year.

One large improvement of the suspension control is implementation of hierarchical control of suspension. After the implementation of this hierarchical control, angular motion of sapphire mirror became very small: less than 100\,nrad in RMS, which is small enough to operate the interferometer. Thanks to this improvement, we succeeded to operate a power-recycling Fabry-Perot Michelson interferometer with the cryogenic payloads.

\paragraph*{\bi Heatlink vibration isolation system}
KAGRA cryogenic payload has heatlinks, which are directly connected to cryocoolers, for cooling sapphire mirror to 20\,K. Since cryocooler makes large vibration compaired to the ground motion at Kamioka site, the vibration through these heatlinks can be contaminate the detector sensitivity of KAGRA. So, we developed heatlink vibration isolation system (HLVIS) to mitigate the vibration through heatlinks.

HLVIS consists of three-stage pendulum. Each stage is suspended by four tension springs in order to mitigate not only holizontal vibration but also vertical vibration. Vibration isolation ratio was measured below 10\,Hz and measurement result was almost matched with designed value.

\paragraph*{\bi New design of mirror inclination control system}
A cryogenic payload has an inclination adjustment system called moving mass system. the moving mass system consists of three components: cryogenic compatible stepper motor, oil-free ball screw, and copper block. We can drive a copper block by rotating stepper motor and change the mass balance of the cryogenic payload to tilt the mirror.

Basically, this moving mass system workes well but in terms of long-term stability, there is an issue to be solved. So, we started to re-design the moving mass system. In the new moving mass system, we adopted pulleys instead of ball screw. In FY 2019, an initial design was finished and prototype was fabricated.

\paragraph*{\bi Sapphire fiber thermal conductivity}
In KAGRA cryogenic payload, high purity aluminum heatlink is installed except for the sapphire mirror to cool the suspension effectively. Instead of this, sapphire mirror is cooled only by sapphire fibers in order not to induce additional vibration and mechanical losses to the mirror. So, thermal conductivity of sapphire fibers directly affects the cooling performance of the cryogenic payload.

In FY2019, we measured thermal conductivity of several sapphire samples and obtained 4000\,W/m$\cdot$K at 20\,K. This value is slightly lower than KAGRA requirement but we started the collaboration with foreign institute for making better sapphire fibers.

\paragraph*{\bi Molecular adosorption at cooled mirror surface}
Gas molecules that hit to cryogenic objects are trapped at these surface. This effect is called cryopumping effect and its pumping power is very large. So, cryogenic mirror of KAGRA can trap residual gas molecules inside vacuum chamber and make thin layers on the mirror surface. Since the thin layers will change the reflectivity of the mirror, this effect can affect the detector sensitivity of KAGRA.
First, we measure the speed of molecular adosorption in the KAGRA cryostat. Then, we calculated the impact for the KAGRA detector sensitivity based on the adosorption speed we measured. In addition to this, we considerred the effective way to remove the adsorbed molecules and demonstrated the disorption by using $CO_2$ laser.

\paragraph*{\bi Doctoral theses}

One doctral thesis, {\it Optical and thermal study of molecular thin layers on cryogenic mirrors in next-generation gravitational wave telescopes} was accepted in FY 2019. 

\paragraph*{\bi Acknowledgement}

Mechanical workshops of Institute for Solid State Physics (The University of Tokyo, Kashiwa campus) and Mechanical Engineering Center of KEK make a large contribution through providing many products for our research.