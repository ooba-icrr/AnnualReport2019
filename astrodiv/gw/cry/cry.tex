\subsubsection*{\bf Cryogenic system}
\noindent {\sf [Spokesperson :\ Takafumi USHIBA]}

\noindent {\sf \small ICRR, The Univ.\ of Tokyo, Kashiwa, Chiba 277-8582}
\vspace{3pt}

A key feature of KAGRA is to cool four sapphire mirrors, which are used for constitution of two arm cavity, to 20\,K. Members working for a cryogenic system, which plays an important role for achieving this unique characteristics, is mainly constituted by that of ICRR, KEK, and the University of Toyama. Here, we summarize the activity of members of ICRR in FY 2017.

\paragraph*{\bi Cryogenic payload}

A KAGRA sapphire mirror is suspended by 9-stage suspension and its bottom 4 stages that include sapphire mirror is called cryogenic payload. The first cryogenic payload with a sapphire mirror was installed at Kamioka in November, 2017. Figure \ref{fig:KAGRAcryo payload} shows the cryogenic payload installed at KAGRA site. This payload was cooled down after installation and reached about 18\,K, which is below the target temperature of KAGRA, 20\,K.

Performance evaluation of sensors and actuators was performed at the site after cooling. It was then confirmed that all sensors and actuators on the cryogenic payload worked even at cryogenic temperature. Damping feedback system for the suspension eigenmode was also implemented, which is significant to operate KAGRA as an interferometer.

\begin{figure}[hbtp]
\begin{center}
\includegraphics[width=0.4\textwidth]{astrodiv/gw/cry/KAGRAcryo_payload.eps}
\caption{\utsm \noindent{\narrower{Cryogenic payload installed into the cryostat at KAGRA site}}}
\label{fig:KAGRAcryo payload}
\end{center}
\end{figure}

\paragraph*{\bi Pure aluminum heat conductor}

In order to cool the mirror down to 20\,K, it is necessary to utilize a high thermal-conductive heat conductor because at cryogenic temperature, especially below 100\,K, thermal radiation becomes small and cannot cool the mirror effectively. However, these heat conductors can easily induce vibration that worsens the KAGRA sensitivity. It is therefore important for the heat conductor to have flexibility enough to reduce the vibration transmission via itself. To realize high thermal conductivity and flexibility, Stranded cables made of pure aluminum of 99.9998\% (5N8 aluminum) are used for KAGRA heat conductor.

In FY 2017, thermal conductivity and flexibility measurement of the heat conductor were performed. Then, we measured a thermal conductivity of $1.85\times10^{4}$\,W/m/K at 10\,K, which is high sufficiently for applying to heat conductors for KAGRA. We also measured a spring constant of stranded heat conductor and compared with that of a single aluminum wire. Then, we confirmed that a spring constant of this stranded aluminum heat conductor is 43 times smaller than that of the single one.

\paragraph*{\bi Vibration measurement inside the cryostat}
We should use huge cryostat (roughly saying, diameter and hight are both 4\,m) in order to install the cryogenic payload and two-layer radiation shields. The resonant frequency of vibration modes of such huge structures is several tens of hertz, which covers KAGRA observation band. So, it is important to make a research for eigenmodes of the cryostat and implement their damping.

An interferometric accelerometer that can be fundamentally used even at the cryogenic temperature was developed. vibration level measurement inside the cryostat under the room temperature condition by using this accelerometer was performed. Then, we found that there is 1000 times lager vibration around 30\,Hz than typical ground vibration of KAGRA site. We also confirmed vibration from the cryocoolers are slightly larger than the ground vibration as well. 

\paragraph*{\bi Magnetic field measurement}

Coil magnet actuators are adopted to control the cryogenic payload, and sapphire mirror has a small magnet on its AR side. So, fluctuation of magnetic field around the payload make some force to the mirror and shake it. Magnetic field of the earth is small enough but there are many devices around the payload such as cryocoolers, vacuum pump, and so on. So, magnetic field generated by these devices should be well studied.

We compared the magnetic field when the cryocoolers are working with that when they are not working. As a result, we confirmed that there are several periodic magnetic field generated by the cryocoolers. Further investigation for these magnetic field issue is promoting.

\paragraph*{\bi Master theses}

Two Master thesis, {\it Vibration Analysis of Cryostat on KAGRA Site} and {\it A Study of the KAGRA Cryogenic Payload System and its Conduction Cooling}, were accepted in FY 2017. 

\paragraph*{\bi Acknowledgement}

Mechanical workshops of Institute for Solid State Physics (The University of Tokyo, Kashiwa campus) and Mechanical Engineering Center of KEK make a large contribution through providing many products for our research.