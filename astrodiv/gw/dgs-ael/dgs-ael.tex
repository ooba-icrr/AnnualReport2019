%!TEX root = ../../../2019main.tex
%%
% main_header.tex
%
%\documentclass[twocolumn]{report} %MM
\documentclass[twocolumn,dvipdfmx]{report} %TS
\usepackage{multicol}

\makeatletter
\newenvironment{tablehere}
  {\def\@captype{table}}
  {}

\newenvironment{figurehere}
  {\def\@captype{figure}}
  {}
\makeatother

%\usepackage[dvips]{graphicx}
%\usepackage[dvipdfmx]{graphicx} %Sako 20180725
\usepackage{graphicx} % Sako comment out
\usepackage{amsmath,amssymb} %YT
\usepackage{times,mathptmx} %YT
\usepackage{url} %TY
\usepackage{subfig} %TY
%\usepackage{./stfloats} %TY11
\usepackage{AnRep}% 
\usepackage{color}
%\usepackage{footnote}
%\makesavenoteenv{table}
\usepackage{threeparttable}
\usepackage{epstopdf}
\usepackage{comment}
\usepackage{seqsplit}
\usepackage{supertabular}

\textwidth 42pc
\textheight 59pc  %%chnged !
\columnsep 1pc
\raggedbottom
\oddsidemargin -2.5pc
\evensidemargin -2.5pc
\topmargin -3pc  %%chnged !
\headsep .8pc
%\mathindent 1pc
\pagestyle{myheadings}

\def\textfraction{0.1} %TY10
\def\floatpagefraction{0.9} %TY10
\def\dblfloatpagefraction{0.9} %TY10

\def\nue{\nu_e}
\def\neb{\bar \nu_e}
\def\num{\nu_\mu}
\def\nmb{\bar \nu_\mu}
\def\nut{\nu_\tau}
\def\ntb{\bar \nu_\tau}
\def\nub{\bar \nu}
\def\dms{\Delta m^2}
\def\sst{\sin^2 2\theta}
\def\tasq{\tan^2\theta}
\def\sisq{\sin^2\theta}

%%% these are bitmap fonts
%%% use PostScript fonts instead these (03/28/2005 Y.Takeuchi)
%\newfont{\sff}{cmssi9} % for affiliation and section italic
%\newfont{\bigsf}{cmss12 scaled 2000} % for title
%\newfont{\midsf}{cmss12 scaled 1000} % for title's subscription
%\newfont{\bfsf}{cmssbx10 scaled 1200} % for title's subscription
%\newfont{\bfsfn}{cmssbx10 scaled 1000} % for title's subscription
%\newfont{\smlsf}{cmss12 scaled 600}  
                       % for section's subscription e.g.$_{\mbox{\smlsf 2}}$
%\newfont{\bigsff}{cmssi12 scaled 2000} % for title's italic

\newfont{\bigsf}{phvr8r scaled 2000} % for title (YT)
\newfont{\bfsf}{phvb8r scaled 1200}  % for title's subscription (YT)


\newcommand{\pmonth}{}
\newcommand{\spec}{}
\newcommand{\no}{}
\newcommand{\bi}{\bfseries\itshape}
\providecommand{\yen}{Y\llap=}

\newcommand{\lsim}{\mbox{\raisebox{-1.ex}
{$\stackrel{\textstyle <}{\textstyle \sim}$}}}
\newcommand{\gsim}{\mbox{\raisebox{-1.ex}
{$\stackrel{\textstyle >}{\textstyle \sim}$}}}
%\newcommand{\sstt}      {\sin^2 2\theta}
%\newcommand{\dms}       {\Delta m^2}
\newcommand{\plumin}[2]{^{+#1}_{-#2}}

\def\nue{\nu_e}
\def\neb{\bar \nu_e}
\def\num{\nu_\mu}
\def\nmb{\bar \nu_\mu}
\def\nut{\nu_\tau}
\def\ntb{\bar \nu_\tau}
\def\nub{\bar \nu}
\def\dms{\Delta m^2}
\def\sst{\sin^2 2\theta}
\def\tasq{\tan^2\theta}

%%%% these are bitmap fonts
%%%% use PS fonts, instead
%\font\lg=cmr12
%\font\bg=cmr17
%\font\sm=cmr7
%\font\fontA=cmr10 scaled \magstep3
%%\font\ssm=cmss12
%\newfont{\ssm}{cmssbx10 scaled 1200} % for title's subscription
%\font\tsmb=cmssbx10 %MM
%\font\ssml=cmss12   %MM
%\font\tsm=cmss10
%\font\utsm=cmss10
%\parindent=10pt

\def\utsm{\sf} %YT
\font\tsmb=phvb8r  %YT
\newfont{\ssm}{phvr8r scaled 1200} %YT
\def\tsm{\sf}  %YT

\def\overset#1\to#2{\mathop{#2}\limits^{#1}}
\def\underset#1\to#2{\mathop{#2}\limits_{#1}}
%\settabs 2 \columns
% A useful Journal macro
\def\Journal#1#2#3#4{{#1} {\bf #2}, #3 (#4)}

% Some useful journal names
\def\NCA{\em Nuovo Cimento}
\def\NIM{\em Nucl. Instrum. Methods}
\def\NIMA{{\em Nucl. Instrum. Methods} A}
\def\NPB{{\em Nucl. Phys.} B}
\def\PLB{{\em Phys. Lett.}  B}
\def\PRL{\em Phys. Rev. Lett.}
\def\PRD{{\em Phys. Rev.} D}
\def\ZPC{{\em Z. Phys.} C}
\def\APJ{\em Ap. J.}
\def\AP{\em Astroparticle Phys.}
\def\JPG{\em J. Phys. G: Nucl. Part. Phys.}

% Some other macros used in the sample text
\def\st{\scriptstyle}
\def\sst{\scriptscriptstyle}
\def\mco{\multicolumn}
\def\epp{\epsilon^{\prime}}
\def\vep{\varepsilon}
\def\ra{\rightarrow}
\def\ppg{\pi^+\pi^-\gamma}
\def\vp{{\bf p}}
\def\ko{K^0}
\def\kb{\bar{K^0}}
\def\al{\alpha}
\def\ab{\bar{\alpha}}
\def\be{\begin{equation}}
\def\ee{\end{equation}}
\def\bea{\begin{eqnarray}}
\def\eea{\end{eqnarray}}
\def\CPbar{\hbox{{\rm CP}\hskip-1.80em{/}}}%temp replacement due to no font

% Extracted from AASTeX (TY)
\newcommand\fdg{\mbox{$.\!\!^\circ$}}%

\renewcommand{\bibname}{AAA}%

%\begin{document}


\vspace{10pt}
\subsubsection*{\bf  Integrated DAQ/control system using real time computers}
\vspace{3pt}
\noindent {\sf [Spokesperson :\ Shoichi OSHINO]}

\vspace{3pt}
\noindent {\sf \small ICRR, The Univ.\ of Tokyo, Hida, Gifu 506-1205}

\vspace{3pt}


The 2019 fiscal year, we started the observation with a power-recycled Fabry-Perot-Michelson interferometer from February 2020. During this observation, we continued to maintain stable DAQ/control system.

\paragraph*{\bi Stable operation with the real time control system}

The first part of 2019 was the process of replacing computers with a faster ones. This is a countermeasure to the glitches that were identified last year, which occur under high loads. Basically the control computers use a real time operating system. Some delay due to the heavy task causes a serious problem for control loops and it emerges as jumps or glitches on many signals. Eventually, by replacing 17 computers, these glitches no longer occur and we are able to control the interferometer with stable digital system.

In the 2018 fiscal year, we used a simple Michelson interferometer to perform operations, but we have to build a more complicated interferometer for the actual observations. Therefore, we have installed various types of hardware to enable more complicated control. In particular, we have installed and enhanced the hardware for length sensing control and output mode cleaner. Finally, we used the digital system of 25 RTPCs, 50 ADCs, 39 DACs and 73 BIOs to control the interferometer.

In late 2019, commissioning work of interferometer began and the sensitivity of gravitational wave is visible. Since the DAQ/control system was already stable, we began reducing the noise to improve sensitivity. A number of circuits have been installed in the KAGRA mine. A number of AC-DC converters are also installed to supply power to these circuits. These DC power supplies generate a lot of magnetic noise, which is expected to affect the sensitivity of the interferometer. We solved this problem by replacing the power supplies farther away from the circuit and wiring the power cables. Most of the power supplies have already been replaced. We will continue to replace the remaining power supplies in fiscal year 2020.




%\end{document}




