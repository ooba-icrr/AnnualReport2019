\subsubsection*{\bf Input and Output Optics}
\noindent {\sf [Spokesperson :\ Keiko KOKEYAMA]}

\vspace{3pt}
\noindent {\sf \small ICRR, The Univ.\ of Tokyo, Hida, Gifu 506-1205}

\vspace{3pt}

The input and output optics of KAGRA consists of the pre-stabilization system for the laser, auxiliary locking system, input optics chain, output optics chain, and detectors. The pre-stabilization system includes the frequency stabilization system, intensity stabilization system, pre-mode cleaner, and modulation system for the main interferometer. The auxiliary locking system includes the phase locking system for the green beam, the fiber system, and the locking system for the arm cavity. The input optics chain includes the input mode cleaner, input Faraday isolator, and input mode matching telescope. The output optics chain includes the output mode matching telescope, output Faraday isolator, and output mode cleaner. The detectors are for the symmetric, antisymmetric, and pick-off ports.

In the fiscal year 2017, major developments towards the bKAGRA phase-1 operation, planned in the spring of 2018, were conducted. The reference cavity was installed on the optical table where the laser source is placed. The reference cavity consists of a cylindrical spacer made of a ultra-low expansion glass, and two mirrors attached on the both side of the cylinder. The length of the cavity is extremely stable, therefore works as a frequency reference for the laser. The frequency of the laser was successfully stabilized, with an acousto-optic modulator installed as the frequency actuator. The input mode cleaner, which had been operating since the iKAGRA operation, was re-commissioned as a part of the frequency stabilization system. At the low frequency, the cavity length of the input mode cleaner is controlled by actuating one of the three mode-cleaner mirrors, whereas the high frequency feedback is applied to the acousto-optic modulator for the frequency stabilization. The cross over frequency of the control was 3 Hz. In addition, the operation procedure was automated. For the intensity stabilization system, the main optical components were installed in the laser room, and the loop was closed. The performance of the intensity stabilization did not reach the goal stability yet, and to be improved in the next fiscal year.

There are two input mode-matching telescopes downstream of the input mode cleaner, where is upstream of the main interferometer. The input mode-matching telescopes are the curved mirrors, hung by the double pendulum, for matching the spacial mode of the beam to the main interferometer mode. The installation of the suspensions had been finished in the previous fiscal year, and in 2017, the fused silica mirrors (diameter 100 mm) were hung from the suspensions. The beam profile after the second input mode-matching telescope was measured, and the beam shape was confirmed to be as designed.

In November, KAGRA hosted the commissioning workshop in Kamioka. Researchers working at the LIGO and VIRGO sites visited the KAGRA site to accelerate the KAGRA experiments. The input and output optics group provided the topics to study in the workshop. Particularly, the noise couplings on the input mode cleaner and laser system were investigated, and useful findings and suggestions were obtained.

The interferometric tilt sensors, developed by Sogang university and tested at the KAGRA site in 2016, was upgraded. The upgraded tilt sensor can sense the orthogonal two degrees of freedom, pitch and yaw motions of a target. The performance was compared with the currently-used sensors (optical levers), and found to be as good in the lower frequencies, and to be better at the frequencies higher than 10 Hz. The tilt sensor was installed permanently at one of the mirrors of the input mode clear.

