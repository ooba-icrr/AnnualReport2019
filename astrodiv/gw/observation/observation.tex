%!TEX root = ../../../2019main.tex
\vspace{10pt}
\subsubsection*{\bf Observation}
\vspace{3pt}
\noindent {\sf [Spokesperson :\ Shinji MIYOKI]}

\vspace{3pt}
\noindent {\sf \small ICRR, The Univ.\ of Tokyo, Hida, Gifu 506-1205}

\vspace{3pt}

After the commissioning as referred in the previous paragraph, KAGRA has finally started its GW observation from 25th February to March 7th with 250 kpc $\sim$ 500 kpc binary range sensitivity for GWs from BNS mergers. After that, another GW network observation with GEO600 gravitational wave telescope was done from 7th April to 21st April with about 500 kpc $\sim$ 800kpc sensitivity. Between these two observations, an additional commissioning was performed to enhance its sensitivity and stability as a telescope. Finally, KAGRA sensitivity reached around 1 Mpc that was one of criteria for KAGRA to join the GW observation network with Adv.LIGO, Adv.Virgo and GEO600. In order to participate in this international GW observation network with Adv.LIGO, Adv.Virgo and GEO600, not only the sensitivity around 1 Mpc binary range, but also many requested criteria were cleared. They are (1) calibration of time domain strain sensitivity, h(t), and its uncertainty budget, (2) preparation of state vector information, (3) rapid response team formation, (4) low-latency KAGRA data transfer to CIT/Virgo, (5) collaboration general computing support, (6) data quality segment database preparation, (7) webpage for IFO status monitor and (8) high-latency data transfer between KAGRA and LV from the October 2019. The latter observation was regarded as ``O3GK'' that were performed according to the LVK MOA that was made in October in 2019. During O3GK, However, Adv.LIGO and Adv.Virgo were offline because of COVID-19 problems. Both observations had engineering runs for a week just before each observation for mainly calibration. Calibration was also done just after observations to evaluate the error range of KAGRA sensitivity. Each observation was operated in three shift system by one operator and one co-operator those were supported by expert members for operational emergency. The duty cycle for O3GK was 53.2\% in science mode and 58.8\% in locked mode. During the online state of both KAGRA and GEO600 during O3GK, an astronomical gamma-ray burst event named ``GRB200415A'' was reported. We are now analyzing our data on this event. 