\twocolumn[
\begin{center}
\vspace{1pc} 
{\bigsf  ASTROPHYSICS AND GRAVITY DIVISION}
\label{astro}
\vspace{20pt}
\end{center}]

%%%%%%%%%%%%%%%%%%%%%%%%%%%%%%%%%%%%%%%%%%%%%%%%%%%%%%%%%%%%%%%%%%%%%%%%%%%%%%
% overview
%%%%%%%%%%%%%%%%%%%%%%%%%%%%%%%%%%%%%%%%%%%%%%%%%%%%%%%%%%%%%%%%%%%%%%%%%%%%%%
\section*{\sf Overview}
\vspace{3pt}
Astrophysics and Gravity Division consists of Gravitational Wave Group, 
The Observational Cosmology Group, Primary Cosmic Ray Group and Theory Group.

The Gravitational Wave Group conducts experimental research of gravitational wave with researchers of gravitational wave experiment and theory in Japan. The main items are the construction of the large scale cryogenic interferometer(KAGRA) at Kamioka underground and the operation of CLIO. For this purpose, KAGRA observatory was established at the beginning of the fiscal year of 2016 to assist the construction of KAGRA gravitational wave telescope. 

%Gravitational Wave Project Office (GWPO) was established at the beginning of the fiscal year of 2011 to assist the construction of KAGRA gravitational wave telescope. The main office is located at Kamioka since 2014. 

The Observational Cosmology Group studies the cosmic history based on deep multi-wavelength observations in collaboration with worldwide researchers. This group has started a new optical deep survey project with the wide-field imager of Hyper Suprime-Cam mounted on the Subaru telescope.

Theory Group conducts both theoretical study of the Universe and astroparticle physics.
\vspace{25pt}
